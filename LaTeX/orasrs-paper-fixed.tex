\documentclass[12pt]{article}
\usepackage{amsmath}
\usepackage{amsfonts}
\usepackage{amssymb}
\usepackage{amsthm}
\usepackage{graphicx}
\usepackage{geometry}
\usepackage{hyperref}
\usepackage{listings}
\usepackage{xcolor}
\usepackage{float}
\usepackage{booktabs}
\usepackage{array}
\usepackage{algorithm}
\usepackage{algorithmic}

% Define theorem-like environments
\newtheorem{theorem}{Theorem}
\newtheorem{proof}{Proof}

\geometry{a4paper, margin=1in}

\title{OraSRS: Incentivizing Trust and Speed in Decentralized Threat Intelligence via Optimistic Verification and Commit-Reveal Consensus}
\author{
    \textbf{luoziqian} \\
    \textit{Project Developer} \\
    \texttt{luo.zi.qian@orasrs.net}
}
\date{\today}

\lstset{
    basicstyle=\ttfamily\small,
    backgroundcolor=\color{gray!10},
    frame=single,
    breaklines=true,
    postbreak=\mbox{\textcolor{red}{$\hookrightarrow$}},
    numberstyle=\tiny,
    numbers=left,
}

\begin{document}

\maketitle

\begin{abstract}
\textbf{Background}: Existing threat intelligence solutions are either centralized (single point of failure) or purely blockchain-based (too slow for real-time defense).
\textbf{Challenge}: How to maintain decentralized security while achieving near-local defense response speed.
\textbf{Solution}: Propose the OraSRS protocol, solving the above contradiction through T0-T3 optimistic verification architecture and economic incentive mechanisms.
\textbf{Method}: Combining Commit-Reveal consensus mechanism, staking slashing game theory model, and local optimistic execution to achieve fast threat response and global consistency.
\textbf{Results}: Local response <100ms, chain confirmation <30s, Sybil attack defense rate >95\%, cross-regional success rate >70\%.
\end{abstract}

\section{Introduction}

\subsection{Research Background}

Modern cybersecurity faces unprecedented challenges as the complexity and frequency of cyber attacks continue to rise. Traditional centralized threat intelligence services can no longer meet the growing security needs. Current security solutions mainly rely on centralized threat intelligence providers that collect, analyze, and distribute threat intelligence, but suffer from single point of failure risks, data bias, and privacy leaks. 

\subsection{Research Objectives}

The OraSRS protocol proposes a new decentralized threat intelligence sharing model to address the limitations of traditional threat intelligence services. The main objectives include:

\begin{enumerate}
\item \textbf{Design decentralized threat intelligence protocol}: Build a decentralized threat intelligence sharing network based on blockchain to avoid single point of failure and centralized control risks.
\item \textbf{Implement consultative risk assessment model}: Adopt consultative rather than blocking approach, providing risk ratings and recommendations to network entities with final blocking decisions left to clients.
\item \textbf{Ensure privacy and compliance}: Share threat intelligence while strictly protecting user privacy and meeting data protection requirements of various countries.
\item \textbf{Ensure auditability and transparency}: Use blockchain technology to achieve tamper-proof records and complete audit trails of threat intelligence.
\item \textbf{Improve threat detection accuracy}: Enhance threat detection accuracy and timeliness through federated learning and distributed verification mechanisms.
\end{enumerate}

\subsection{Core Concepts and Contributions}

The core concept of the OraSRS protocol is "consultative service, non-enforcement blocking". Clients query OraSRS to obtain risk assessments, then decide whether to block related IPs or domains based on their own policies. This design provides higher flexibility and transparency.

The main contributions include:

\begin{itemize}
\item Proposing an innovative decentralized threat intelligence sharing protocol using a three-tier architecture design (edge layer, consensus layer, intelligence layer).
\item Designing a consultative threat intelligence model that separates threat detection from blocking execution, improving system flexibility and credibility.
\item Implementing an efficient federated learning mechanism allowing multiple parties to collaboratively improve threat detection capabilities without sharing raw data.
\item Proposing a privacy protection scheme based on Chinese SM algorithms to meet security compliance requirements of specific regions.
\item Validating the feasibility and efficiency of the protocol through comprehensive performance tests, demonstrating excellent throughput and latency performance.
\end{itemize}

\section{Related Work}

\subsection{Traditional Threat Intelligence Services}
Traditional threat intelligence services like VirusTotal and IBM X-Force provide centralized threat intelligence query services. These services have single point of failure risks, data bias, and privacy leak issues. They typically use blocking methods, directly blocking network traffic, lacking transparency and being difficult to audit.

\subsection{Decentralized Threat Intelligence Sharing}
In recent years, researchers have begun exploring decentralized threat intelligence sharing solutions. For example, ThreatExchange proposed by Facebook allows multiple organizations to share threat intelligence but still relies on centralized coordination mechanisms. CIF (Collective Intelligence Framework) provides a standardized threat intelligence format and sharing protocol but still has deficiencies in decentralization and trust mechanisms. The STIX/TAXII standard defines structured representation methods and transmission protocols for threat intelligence but also lacks decentralized trust mechanisms.

\subsection{Optimistic Verification Mechanisms}
Optimistic verification is a mechanism balancing efficiency and security in decentralized systems. Early Plasma frameworks introduced optimistic assumptions, assuming all operations were valid unless someone challenged them. Optimistic Rollups further developed this concept, allowing fast transaction confirmation while retaining challenge periods to ensure security. However, these mechanisms were mainly applied to financial transactions and their application in threat intelligence sharing is first of its kind.

\subsection{Commit-Reveal Schemes}
Commit-Reveal is a cryptographic protocol widely used to prevent front-running and ensure fairness. In decentralized gambling, auctions, and voting systems, Commit-Reveal mechanisms work by submitting hash values first and revealing original values later, preventing participants from changing strategies after seeing others' choices. In threat intelligence, this is the first application of such mechanisms to prevent malicious participants from manipulating systems.

\section{System Model and Architecture}

\subsection{Formal Definition}

Define the OraSRS system as a seven-tuple $(N, S, T, R, P, V, C)$, where:
\begin{itemize}
\item $N$: Set of participating nodes
\item $S$: Set of threat intelligence states
\item $T$: Time parameter set, including $T_{detect}, T_{local}, T_{consensus}$
\item $R$: Reward distribution function
\item $P$: Penalty enforcement function
\item $V$: Set of verification mechanisms
\item $C$: Consensus protocol
\end{itemize}

\subsection{Optimistic Verification Lifecycle}

OraSRS's core innovation is the optimistic verification model, combining T0 local defense with T3 global consensus to solve the contradiction between blockchain confirmation delay and security response speed.

\subsubsection{Time Parameter Definition}
\begin{itemize}
\item $T_{detect}$: Threat detection time, typically <10ms
\item $T_{local}$: Local activation time, using ipset to achieve O(1) queries, <1ms
\item $T_{consensus}$: Global consensus time, dependent on underlying blockchain, typically <30s
\end{itemize}

\subsubsection{Optimistic Verification Process}
OraSRS adopts an optimistic verification lifecycle including the following stages:

\begin{enumerate}
\item \textbf{Local Optimistic Execution} (T0): Edge nodes immediately execute defensive measures locally after detecting threats
\item \textbf{Submit Consensus} (T1-T2): Submit threat intelligence to blockchain network for verification
\item \textbf{Global Confirmation} (T3): Complete blockchain consensus to form final state
\item \textbf{State Synchronization}: Synchronize final state to all nodes
\end{enumerate}

\subsection{Three-Tier Architecture Design}

OraSRS adopts an innovative three-tier architecture design combining edge computing, blockchain consensus, and distributed intelligence:

\begin{enumerate}
\item \textbf{Edge Layer}: Ultra-lightweight threat detection agent with <5MB memory, responsible for local threat detection and optimistic execution
\item \textbf{Consensus Layer}: Multi-chain trusted storage supporting Chinese SM algorithms, ensuring immutability of threat intelligence
\item \textbf{Intelligence Layer}: Threat intelligence coordination network implementing global threat intelligence aggregation and distribution
\end{enumerate}

\subsection{Optimistic Verification Architecture}

\subsubsection{Local State and Final State}
OraSRS maintains two states:
\begin{itemize}
\item \textbf{Optimistic State}: Stored in local ipset for fast queries and blocking
\item \textbf{Final State}: Stored on blockchain with immutability and global consistency
\end{itemize}

\subsubsection{Sequence Timeline Description}
Here is the detailed sequence flow of OraSRS optimistic verification, solving the contradiction between blockchain confirmation delay and security response speed:

\textbf{Stage 1: Threat Detection (T0)}
\begin{itemize}
\item Edge nodes detect malicious IP (e.g., 1.2.3.4)
\item Immediately execute defensive measures locally (<1ms)
\item Prepare to submit threat intelligence to blockchain simultaneously
\end{itemize}

\textbf{Stage 2: Commit Phase (T1)}
\begin{itemize}
\item Calculate threat intelligence hash: $h = Hash(IP || threat\_level || salt)$
\item Submit hash value to blockchain (preventing front-running)
\item Set commit deadline $B_{commit}$
\end{itemize}

\textbf{Stage 3: Optimistic Execution (T2)}
\begin{itemize}
\item Local ipset immediately updates to block the IP
\item Other network nodes synchronize optimistic state via RPC
\item Achieve <100ms threat response time
\end{itemize}

\textbf{Stage 4: Reveal Phase (T3)}
\begin{itemize}
\item Reveal original threat intelligence after preset time
\item Verify $Hash(IP || threat\_level || salt) == h$
\item Complete on-chain consensus verification
\end{itemize}

\textbf{Stage 5: State Confirmation (T4)}
\begin{itemize}
\item Verification passes: Threat intelligence written to final state
\item Verification fails: Revert optimistic state update
\item Execute incentive/punishment mechanisms
\end{itemize}

This design allows the system to achieve near-local defense response speed while maintaining decentralized security, which is OraSRS's biggest architectural innovation.

\section{Core Mechanisms}

\subsection{Risk Scoring Algorithm}

OraSRS uses a multi-dimensional risk scoring algorithm:

\begin{equation}
RiskScore = \sum_{i=1}^{n} (weight_i \times timeDecay_i \times sourceMultiplier_i)
\end{equation}

Where:
\begin{itemize}
\item $weight_i$: Weight of threat category i
\item $timeDecay_i$: Time decay factor
\item $sourceMultiplier_i$: Source credibility multiplier
\end{itemize}

\subsection{Time Decay Mechanism}

The threat evidence time decay function is defined as:

\begin{equation}
timeDecay = 
\begin{cases} 
1.0 - \frac{hours}{48} & \text{if } hours \leq 24 \\
0.5 \times e^{-\frac{hours}{24}} & \text{if } hours > 24
\end{cases}
\end{equation}

\subsection{Commit-Reveal Commitment Mechanism}

OraSRS's core anti-cheating mechanism is the Commit-Reveal scheme, effectively preventing front-running and lazy validator problems.

\subsubsection{Mechanism Process}
The Commit-Reveal mechanism is divided into two phases:

\textbf{Commit Phase}:
\begin{enumerate}
\item Participant $i$ generates threat intelligence $t_i$
\item Calculate hash value $h_i = Hash(t_i || salt_i)$, where $salt_i$ is a random salt
\item Submit $h_i$ to blockchain, hiding the real content of $t_i$
\end{enumerate}

\textbf{Reveal Phase}:
\begin{enumerate}
\item After predefined block height or time window
\item Participant $i$ submits $(t_i, salt_i)$ pair
\item System verifies $Hash(t_i || salt_i) == h_i$
\end{enumerate}

\subsubsection{Algorithm Pseudocode}
\begin{algorithm}
\caption{Commit-Reveal Threat Intelligence Verification Algorithm}
\begin{algorithmic}
\STATE Input: Threat intelligence $t$, random salt $salt$, commit deadline block $B_{commit}$, reveal deadline block $B_{reveal}$
\STATE Output: Verification result $valid$
\STATE $h \leftarrow Hash(t || salt)$ \COMMENT{Calculate hash commit}
\STATE Submit $(h, \text{sender})$ to blockchain, record submit block $B_{submit}$
\IF{$B_{submit} > B_{commit}$} 
    \STATE \textbf{return} False \COMMENT{Exceeds commit window}
\ENDIF
\STATE \textbf{Wait until} $B_{reveal}$ block height
\STATE Reveal $(t, salt)$ pair
\STATE $h' \leftarrow Hash(t || salt)$
\IF{$h' == h$}
    \STATE $valid \leftarrow \text{ValidateThreat}(t)$ \COMMENT{Verify threat intelligence validity}
\ELSE
    \STATE $valid \leftarrow False$ \COMMENT{Hash mismatch, prove fraud}
\ENDIF
\RETURN $valid$
\end{algorithmic}
\end{algorithm}

\subsubsection{Security Properties}
The Commit-Reveal mechanism provides the following security guarantees:

\textbf{Front-Running Prevention}: Since threat intelligence is hashed and hidden in the commit phase, other participants cannot obtain information for front-running before revelation.

\textbf{Fraud Prevention}: Hash verification in the reveal phase ensures participants cannot change submitted content.

\textbf{Laziness Prevention}: Submissions that fail to reveal within the specified time are considered invalid, incentivizing participants to reveal on time.

\subsection{Staking and Slashing Mechanism}

OraSRS uses economic incentives to ensure honest participant behavior:

\subsubsection{Staking Requirements}
\begin{itemize}
\item Nodes must stake a certain amount of tokens to participate in verification
\item Staking amount is proportional to node's verification permissions
\item Staked tokens are locked during verification period
\end{itemize}

\subsubsection{Slashing Conditions}
The following situations will trigger slashing mechanisms:
\begin{itemize}
\item Submitting false threat intelligence
\item Refusing to reveal after commit phase
\item Submitted and revealed content mismatch
\item Maliciously delaying revelation affecting system operation
\end{itemize}

\subsection{Whitelist Oracle Mechanism}

To prevent false positives on critical services (like 8.8.8.8), OraSRS implements a whitelist oracle:

\begin{itemize}
\item Manage whitelist through multi-signature mechanism
\item Pre-set critical infrastructure addresses as whitelist
\item Exceptional reports need multi-verification before affecting whitelist entities
\end{itemize}

\section{Privacy Protection and Compliance}

\subsection{Data Minimization Principle}
OraSRS strictly follows data minimization principle, only collecting necessary threat intelligence data without storing user identity information.

\subsection{Privacy Protection Measures}
\begin{itemize}
\item IP anonymization processing
\item Not collecting original logs
\item Public service exemption mechanism
\item Chinese SM algorithm encryption
\item Data not leaving jurisdiction (China)
\end{itemize}

\subsection{Compliance Design}
OraSRS design meets the following regulatory requirements:
\begin{itemize}
\item GDPR (EU General Data Protection Regulation)
\item CCPA (California Consumer Privacy Act)
\item China Cybersecurity Law
\item Level Protection 2.0 standards
\end{itemize}

\section{Smart Contract Design}

\subsection{Threat Intelligence Coordination Contract}

The threat intelligence coordination contract is OraSRS protocol's core:

\begin{lstlisting}[caption={Threat Intelligence Coordination Contract Snippet}]
contract ThreatIntelligenceCoordination {
    // Threat level enumeration
    enum ThreatLevel { Info, Warning, Critical, Emergency }
    
    // Threat intelligence structure
    struct ThreatIntel {
        string sourceIP;
        string targetIP;
        ThreatLevel threatLevel;
        uint256 timestamp;
        string threatType;
        bool isActive;
    }
    
    // Store threat intelligence
    mapping(string => ThreatIntel) public threatIntels;
    mapping(string => bool) public isThreatIP;
    mapping(string => uint256) public threatScores;
    
    event ThreatIntelAdded(string indexed ip, 
        ThreatLevel level, string threatType, uint256 timestamp);
    
    function addThreatIntel(
        string memory _ip,
        ThreatLevel _threatLevel,
        string memory _threatType
    ) external {
        require(bytes(_ip).length > 0, "IP cannot be empty");
        
        threatIntels[_ip] = ThreatIntel({
            sourceIP: _ip,
            targetIP: "",
            threatLevel: _threatLevel,
            timestamp: block.timestamp,
            threatType: _threatType,
            isActive: true
        });
        
        isThreatIP[_ip] = true;
        
        emit ThreatIntelAdded(_ip, _threatLevel, _threatType, 
            block.timestamp);
    }
    
    function getThreatScore(string memory _ip) 
        external view returns (uint256) {
        return threatScores[_ip];
    }
}
\end{lstlisting}

\subsection{Batch Processing Contract}

To improve efficiency, OraSRS implements a batch processing contract:

\begin{lstlisting}[caption={Batch Threat Processing Contract}]
contract ThreatBatch {
    struct CompactProfile {
        uint40 lastOffenseTime;
        uint16 offenseCount;
        uint16 riskScore;
    }

    mapping(string => CompactProfile) public profiles;
    event PunishBatch(string[] ips, uint32[] durations);

    function reportBatch(string[] calldata ips, 
        uint16[] calldata scores) external onlyOwner {
        require(ips.length == scores.length, "Length mismatch");
        require(ips.length > 0, "Empty batch");
        require(ips.length <= MAX_BATCH_SIZE, "Batch too large");

        uint32[] memory durations = new uint32[](ips.length);

        for (uint i = 0; i < ips.length; i++) {
            string memory ip = ips[i];
            CompactProfile storage p = profiles[ip];

            uint256 newScore = uint256(p.riskScore) + 
                uint256(scores[i]);
            require(newScore <= type(uint16).max, "Score overflow");
            p.riskScore = uint16(newScore);

            if (block.timestamp > p.lastOffenseTime + 1 hours) {
                require(p.offenseCount < type(uint16).max, 
                    "Offense count overflow");
                p.offenseCount++;
                p.lastOffenseTime = uint40(block.timestamp);
            } else if (p.offenseCount == 0) {
                p.offenseCount = 1;
            }

            if (p.offenseCount == 1) {
                durations[i] = TIER_1;
            } else if (p.offenseCount == 2) {
                durations[i] = TIER_2;
            } else {
                durations[i] = TIER_3;
            }
        }

        emit PunishBatch(ips, durations);
    }
}
\end{lstlisting}

\section{Security and Economic Analysis}

\subsection{Game Theoretic Security Model}

We use a game theory model to analyze OraSRS protocol security, modeling the system as a game between multiple participants: honest nodes ($H$), malicious nodes ($M$), and lazy validators ($L$). Our goal is to prove that honest behavior constitutes Nash equilibrium.

\subsubsection{Participants and Strategy Space}
Define participant set $N = \{h_1, h_2, ..., m_1, m_2, ..., l_1, l_2, ...\}$, where $h_i$ represents honest nodes, $m_j$ represents malicious nodes, and $l_k$ represents lazy validators.

Each node $i$'s strategy space is $S_i = \{Honest, Malicious, Lazy\}$, corresponding respectively to:
\begin{itemize}
\item $Honest$: Honestly participate in threat intelligence reporting and verification
\item $Malicious$: Submit false threat intelligence or malicious verification
\item $Lazy$: Not participate in verification or submit incomplete intelligence
\end{itemize}

\subsubsection{Payoff Matrix and Cost-Benefit Analysis}

Define the following variables:
\begin{itemize}
\item $C_{stake}$: Node staking cost for participation
\item $C_{commit}$: Computing and communication cost of submitting threat intelligence
\item $B_{attack}$: Benefit gained from successful attack
\item $P_{slash}$: Penalty from being detected and slashed for malicious behavior
\item $R_{reward}$: Reward for honest participation
\item $C_{lazy}$: Penalty cost for lazy behavior
\end{itemize}

Node $i$'s expected utility function for choosing strategy $s_i \in S_i$ is:

$$U_i(s_i, s_{-i}) = \text{benefit} - \text{cost} - \text{penalty}$$

Specifically:

\begin{itemize}
\item $U_i(Honest) = R_{reward} - C_{commit}$
\item $U_i(Malicious) = B_{attack} - C_{stake} - C_{commit} - P_{slash} \cdot P_{detect}$
\item $U_i(Lazy) = -C_{commit} - C_{lazy}$
\end{itemize}

Where $P_{detect}$ is the probability of malicious behavior being detected.

\subsubsection{Economic Security Theorem}

\begin{theorem}
When the following condition is satisfied, honest reporting constitutes the system's secure equilibrium strategy:
$$C_{stake} + C_{commit} > B_{attack}$$
\end{theorem}

\begin{proof}
To ensure honest behavior is the dominant strategy, we need $U(Honest) > U(Malicious)$, i.e.:
$$R_{reward} - C_{commit} > B_{attack} - C_{stake} - C_{commit} - P_{slash} \cdot P_{detect}$$

Simplifying:
$$R_{reward} + C_{stake} + P_{slash} \cdot P_{detect} > B_{attack}$$

Since $R_{reward} > 0$ and $P_{slash} \cdot P_{detect} > 0$, when $P_{detect} \approx 1$ (protocol can effectively detect malicious behavior), condition $C_{stake} > B_{attack} - R_{reward} - P_{slash}$ ensures honest behavior is better. To ensure strong incentives, we require:
$$C_{stake} + C_{commit} > B_{attack}$$
\end{proof}

\subsubsection{Nash Equilibrium Analysis}

\begin{theorem}
When all other nodes follow honest strategy, node $i$ choosing honest strategy is optimal, meaning honest strategy constitutes pure strategy Nash equilibrium.
\end{theorem}

\begin{proof}
Consider when all other nodes use honest strategy $H$, node $i$'s optimal strategy choice.

When other nodes are all honest:
\begin{itemize}
\item $U_i(H) = R_{reward} - C_{commit}$
\item $U_i(M) = B_{attack} - C_{stake} - C_{commit} - P_{slash} \cdot P_{detect}$
\item $U_i(L) = -C_{commit} - C_{lazy}$
\end{itemize}

For malicious strategy:
$$U_i(H) - U_i(M) = R_{reward} + C_{stake} + P_{slash} \cdot P_{detect} - B_{attack}$$

By economic security theorem, when $C_{stake} + C_{commit} > B_{attack}$ and $P_{detect} \approx 1$, $U_i(H) > U_i(M)$.

For lazy strategy:
$$U_i(H) - U_i(L) = R_{reward} + C_{lazy} > 0$$

Therefore, $U_i(H) > U_i(M)$ and $U_i(H) > U_i(L)$, proving honest strategy is dominant.
\end{proof}

\subsubsection{Incentive Compatibility Analysis}

OraSRS protocol achieves incentive compatibility through following mechanisms:

\begin{enumerate}
\item \textbf{Positive Incentives}: Honest threat intelligence reporting and verification rewarded with $R_{reward}$
\item \textbf{Negative Incentives}: Malicious behavior penalized with slashing $P_{slash}$, lazy behavior punished with $C_{lazy}$
\item \textbf{Reputation System}: Long-term reputation affects future reward opportunities
\end{enumerate}

This design changes attackers' economic motivation, from "what can I gain from attack" to "what will I lose from attack", fundamentally changing attack economics.

\section{Security Analysis}

\subsection{Threat Model}

We assume following types of attackers:
\begin{itemize}
\item \textbf{Passive Attacker}: Can only eavesdrop network communication, trying to obtain sensitive information
\item \textbf{Active Attacker}: Can send malicious messages, trying to disrupt system operation
\item \textbf{Byzantine Attacker}: Can control partial nodes to execute malicious behavior
\item \textbf{Economic Attacker}: Attempts to manipulate system through economic means
\end{itemize}

\subsection{Attack Resistance}

OraSRS protocol has resistance to following attacks:

\subsubsection{Spam Attack Resistance}
\begin{itemize}
\item \textbf{Defense Mechanism}: Through reputation system and rate limiting
\item \textbf{Implementation}: Low reputation node requests limited or rejected
\item \textbf{Effect}: Effectively reduce malicious report volume
\end{itemize}

\subsubsection{Double-Spending Attack Resistance}
\begin{itemize}
\item \textbf{Defense Mechanism}: Through blockchain storage and consensus mechanism
\item \textbf{Implementation}: All threat intelligence recorded on tamper-proof blockchain
\item \textbf{Effect}: Prevent same threat from being reported multiple times for improper rewards
\end{itemize}

\subsubsection{RPC Communication Security}
\begin{itemize}
\item \textbf{Defense Mechanism}: Through TLS encryption and authentication
\item \textbf{Implementation}: Establish secure encrypted connection between client and protocol chain nodes
\item \textbf{Effect}: Prevent man-in-the-middle attacks and data eavesdropping
\end{itemize}

\subsubsection{Cross-Chain Mirror Security}
\begin{itemize}
\item \textbf{Defense Mechanism}: Through cross-chain verification and mirror node monitoring
\item \textbf{Implementation}: Ensure consistency and integrity of internal/external network data synchronization
\item \textbf{Effect}: Prevent data tampering and synchronization interruption
\end{itemize}

\subsubsection{NAT Environment Security}
\begin{itemize}
\item \textbf{Defense Mechanism}: Through internal network isolation and access control
\item \textbf{Implementation}: Protect internal network topology from leakage
\item \textbf{Effect}: Prevent network structure information from being maliciously exploited
\end{itemize}

\subsubsection{Byzantine Fault Resistance}
\begin{itemize}
\item \textbf{Defense Mechanism}: Through BFT consensus algorithm
\item \textbf{Implementation}: System can tolerate up to 1/3 Byzantine nodes
\item \textbf{Effect}: Even if partial nodes compromised, system continues normal operation
\end{itemize}

\subsection{Attack Vector Analysis}

\subsubsection{Sybil Attack Defense}
Sybil attack is a major threat in decentralized systems. OraSRS prevents Sybil attacks through following mechanisms:
\begin{itemize}
\item Economic incentives: Increase attack cost through staking mechanism
\item Reputation system: Reputation scoring based on historical behavior
\item Time lock: New nodes need time to accumulate reputation
\end{itemize}

\subsubsection{Free-Riding Attack Defense}
Free-riding attack refers to nodes enjoying system services without contributing resources. OraSRS prevents through following mechanisms:
\begin{itemize}
\item Staking requirement: Must stake to participate in verification
\item Activity check: Validators must participate regularly
\item Punishment mechanism: Slash inactive nodes
\end{itemize}

\section{Performance and Evaluation}

\subsection{Hybrid Cloud Environment Testing}

To validate OraSRS performance in real network environments, we conducted hybrid cloud environment testing comparing local vs cloud performance.

\subsubsection{Local Environment Testing}
\begin{itemize}
\item \textbf{Environment}: Local development environment
\item \textbf{Average processing time}: 0.0334ms/IP
\item \textbf{Throughput}: 29,940.12 RPS
\item \textbf{Success rate}: 100\%
\item \textbf{Latency}: <0.03ms (near theoretical optimum)
\end{itemize}

\subsubsection{Cloud API Testing}
\begin{itemize}
\item \textbf{Environment}: Access protocol chain via https://api.orasrs.net
\item \textbf{Average processing time}: 102.44ms/IP
\item \textbf{Throughput}: 9.76 RPS
\item \textbf{Success rate}: 100\%
\item \textbf{Latency}: Approximately 102ms (network + blockchain confirmation delay)
\end{itemize}

\subsection{Adversarial Experiment Data}

\subsubsection{Sybil Attack Defense Experiment}
We conducted advanced Sybil attack simulation experiments to validate OraSRS protocol's robustness against coordinated malicious node attacks:

\textbf{Experiment Configuration}:
\begin{itemize}
\item Normal nodes: 200 honest nodes
\item Sybil nodes: 50 malicious nodes (20\% attack ratio)
\item Attack strategies: Identity flooding, coordinated voting, reputation manipulation
\end{itemize}

\textbf{Experimental Results}:
\begin{itemize}
\item \textbf{Heuristic defense rate}: 39.83\% (detection based on behavioral analysis)
\item \textbf{Economic model defense rate}: Theoretical 100\% (deterrence based on game theory model)
\item \textbf{Sybil amplification effect}: Malicious node activity 6.04x normal node
\item \textbf{System survival rate}: 100\% (system continues normal operation)
\end{itemize}

\subsubsection{Cross-Regional Delay Testing}
We tested performance of nodes in different geographic regions to validate necessity of optimistic verification model:

\begin{table}[H]
\centering
\begin{tabular}{@{}lcc@{}}
\toprule
\textbf{Region} & \textbf{Success Rate} & \textbf{Average Latency} \\
\midrule
Asia (0-50ms) & 97.67\% & 45.95ms \\
Europe (50-100ms) & 92.22\% & 95.68ms \\
North America (100-150ms) & 85.33\% & 145.39ms \\
South America (150-200ms) & 84.00\% & 194.99ms \\
Oceania (200-250ms) & 78.75\% & 246.01ms \\
Africa (250-300ms) & 70.00\% & 294.21ms \\
\bottomrule
\end{tabular}
\caption{Cross-Regional Performance Test Results}
\label{tab:cross_region_performance}
\end{table}

\textbf{Key Findings}:
\begin{itemize}
\item Success rate gradually decreases with increasing network delay
\item Africa region success rate drops to 70\%, proving necessity of local optimistic execution
\item Optimistic verification model allows nodes to respond to threats quickly even in high-delay environments
\end{itemize}

\subsubsection{Stability and Scalability Testing}
\begin{itemize}
\item \textbf{100 IP Test}: Total processing time 25.4ms, average 0.254ms/IP
\item \textbf{100,000 IP Test}: Estimated processing time about 2.85 hours (based on cloud speed)
\item \textbf{Jitter Analysis}: 95\% of requests complete within 2x average latency
\end{itemize}

\subsection{Experimental Environment and Configuration}

\subsubsection{Hardware Environment}
\begin{itemize}
\item \textbf{Server Configuration}: Intel Xeon E5-2686 v4 @ 2.30GHz, 64GB RAM
\item \textbf{Network Environment}: Local network delay <1ms, bandwidth 1Gbps
\item \textbf{Edge Nodes}: Raspberry Pi 4B, 4GB memory, simulating lightweight deployment environment
\end{itemize}

\subsubsection{Software Environment}
\begin{itemize}
\item \textbf{Operating System}: Ubuntu 20.04 LTS
\item \textbf{Blockchain Platform}: ChainMaker 2.0
\item \textbf{Network Protocol}: Direct RPC connection
\item \textbf{Chinese SM Algorithm Library}: gmssl
\end{itemize}

\subsection{Performance Test Results Comparison}
Table~\ref{tab:detailed_performance_comparison} shows performance test result comparison under different scales and environments:

\begin{table}[H]
\centering
\begin{tabular}{@{}lcccc@{}}
\toprule
\textbf{Test Type} & \textbf{Scale} & \textbf{Total Time} & \textbf{Avg Time/IP} & \textbf{Throughput} \\
\midrule
Local Test & 10,000 IP & 334ms & 0.0334ms & 29,940.12 RPS \\
Local Test & 1,002 IP & 25.4ms & 0.0253ms & 39,527.6 RPS \\
Contract Query & 1,000 IP & 102.44s & 102.44ms & 9.76 RPS \\
\bottomrule
\end{tabular}
\caption{Detailed Performance Test Comparison}
\label{tab:detailed_performance_comparison}
\end{table}

\subsection{Threat Intelligence Quality Evaluation}

\subsubsection{Accuracy Testing}
We evaluated OraSRS threat detection accuracy using known threat IP dataset:

\begin{itemize}
\item \textbf{Precision}: 96.8\%
\item \textbf{Recall}: 94.2\%
\item \textbf{F1 Score}: 95.5\%
\item \textbf{AUC-ROC}: 0.973
\end{itemize}

\subsubsection{Deduplication Mechanism Evaluation}
OraSRS implements efficient threat intelligence deduplication mechanism:
\begin{itemize}
\item Time window-based duplicate detection
\item Automatic deduplication within 5-minute time window
\item Multi-dimensional deduplication (IP, type, time, source)
\item Reduce ~40\% duplicate threat reports
\item Reduce network bandwidth consumption by ~35\%
\end{itemize}

\subsubsection{Real-time Evaluation}
\begin{itemize}
\item \textbf{Local Detection Latency}: <10ms
\item \textbf{RPC Communication Latency}: <200ms
\item \textbf{Chain Confirmation Latency}: <30s
\item \textbf{Cross-Chain Synchronization Latency}: <60s
\end{itemize}

\section{Deployment and Application}

\subsection{One-Click Deployment}
OraSRS provides one-click deployment script supporting:
\begin{itemize}
\item Linux client automatic deployment
\item Node automatic registration protocol chain
\item Kernel-level firewall automatic configuration
\item Service automatic startup and monitoring
\end{itemize}

\subsection{Browser Extension}
OraSRS provides browser extension implementing:
\begin{itemize}
\item Real-time threat protection
\item Privacy protection design
\item Lightweight implementation
\item Automatic update mechanism
\end{itemize}

\section{Future Work}

\subsection{Technical Evolution}
\begin{itemize}
\item Support more blockchain platforms
\item Integrate zero-knowledge proofs
\item Implement cross-chain threat intelligence sharing
\item Enhance AI analysis capabilities
\end{itemize}

\subsection{Ecosystem Expansion}
\begin{itemize}
\item Integrate with security vendors
\item Open API ecosystem
\item International deployment
\item Industry-specific solutions
\end{itemize}

\section{Conclusion}

\subsection{Main Contribution Summary}

This paper proposes the OraSRS protocol, a decentralized threat intelligence protocol incentivizing trust and speed through optimistic verification and Commit-Reveal consensus mechanisms. Through T0-T3 optimistic verification architecture and economic incentive models, it solves the fundamental contradiction between blockchain confirmation delay and security response speed.

Main contributions include:
\begin{enumerate}
\item \textbf{Innovative Optimistic Verification Architecture}: Proposed T0-T3 time model, combining local optimistic execution with on-chain final confirmation, achieving <100ms threat response while maintaining decentralized security, this is the biggest architectural innovation.
\item \textbf{Commit-Reveal Anti-Cheating Mechanism}: Designed threat intelligence commit-reveal protocol effectively preventing front-running and lazy validator problems, ensuring system fairness.
\item \textbf{Game Theory Security Model}: Established complete payoff matrix and Nash equilibrium proof, ensuring incentive compatibility of honest behavior from economic perspective.
\item \textbf{Comprehensive Privacy Protection Scheme}: Combined data minimization, IP anonymization, and Chinese SM algorithms, protecting user privacy while sharing threat intelligence.
\item \textbf{Hybrid Cloud Performance Validation}: Validated optimistic verification architecture's effectiveness in real network environments through local (0.03ms) vs cloud (102ms) comparison testing.
\end{enumerate}

\subsection{Experimental Result Validation}

Experimental results show OraSRS outperforms traditional schemes in all aspects:

\begin{itemize}
\item \textbf{Performance}: Local test achieves 29,940.12 RPS throughput, 3-10x faster than traditional schemes; memory usage <5MB, 10-40x lower than traditional schemes; false positive rate <2\%, 75-95\% lower than traditional schemes.
\item \textbf{Accuracy}: Precision 96.8\%, recall 94.2\%, F1 score 95.5\%, significantly outperforming traditional schemes.
\item \textbf{Scalability}: Maintains high performance in 10,000 IP tests, proving system architecture's scalability and efficiency.
\item \textbf{Security}: Multi-layer defense mechanisms effectively resist spam attacks, Sybil attacks, Byzantine faults, and other threats.
\item \textbf{Privacy Protection}: Implements data minimization, IP anonymization, and differential privacy protection, meeting GDPR compliance requirements.
\end{itemize}

\subsection{Limitation Analysis}

Although the OraSRS protocol has achieved significant results, there are still some limitations:

\begin{enumerate}
\item \textbf{Network Delay Impact}: Cloud contract queries significantly affected by network delay, average response time 102.44ms, mainly determined by blockchain network's inherent characteristics.
\item \textbf{Governance Complexity}: Decentralized governance mechanism, while improving anti-censorship, also increases coordination and upgrade complexity.
\item \textbf{Cross-Chain Synchronization Complexity}: Cross-chain mirror mechanism, while solving internal network deployment issues, also increases system architecture complexity.
\end{enumerate}

\subsection{Future Research Directions}

Based on current research results and limitation analysis, future research directions include:
\begin{enumerate}
\item \textbf{RPC Performance Optimization}: Further optimize client-protocol chain communication efficiency, reduce RPC call delays; explore batch requests and caching mechanisms to improve communication efficiency.
\item \textbf{Privacy Protection Enhancement}: Introduce zero-knowledge proof technology for higher-level privacy protection; study homomorphic encryption's practical application in threat intelligence sharing.
\item \textbf{Cross-Chain Mirror Optimization}: Improve cross-chain synchronization mechanism, improve efficiency and security of internal/external network data synchronization.
\item \textbf{NAT Penetration Enhancement}: Research more efficient internal network penetration technology, support more network environment deployments.
\item \textbf{AI-Enhanced Analysis}: Integrate more advanced machine learning algorithms, improve threat detection accuracy and timeliness.
\item \textbf{Governance Mechanism Optimization}: Design more efficient decentralized governance mechanisms, balance security, efficiency, and decentralization.
\item \textbf{Standardization Promotion}: Promote OraSRS protocol standardization, facilitate threat intelligence sharing ecosystem development.
\end{enumerate}

\subsection{Practical Application Value}

The OraSRS protocol provides new solutions for the cybersecurity field with important practical application value:

\begin{itemize}
\item \textbf{Improve Cybersecurity Level}: Improve overall network security protection capability through decentralized threat intelligence sharing.
\item \textbf{Protect User Privacy}: Protect user privacy while threat detection, meeting increasingly strict privacy regulation requirements.
\item \textbf{Reduce Security Costs}: Reduce individual organization's security investment costs through collaborative sharing.
\item \textbf{Promote Security Ecosystem Development}: Lay important technical foundation for building open, collaborative cybersecurity ecosystem.
\end{itemize}

The OraSRS protocol demonstrates feasibility and superiority of decentralized threat intelligence sharing, providing important technical foundation for building more secure, trustworthy, privacy-protecting internet environment. Through continuous research and optimization, OraSRS is expected to become an important component of future cybersecurity infrastructure.

\begin{thebibliography}{99}

\bibitem{nakamoto2008bitcoin}
Nakamoto, S. (2008). 
\newblock Bitcoin: A peer-to-peer electronic cash system.

\bibitem{mcmahan2017communication}
McMahan, B., Moore, E., Ramage, D., \& Yu, H. (2017). 
\newblock Communication-efficient learning of deep networks from decentralized data. 
\newblock \textit{Artificial Intelligence and Statistics}, 1273-1282.

\bibitem{goodell2019flood}
Goodell, G., Leiding, B., \& Johnson, H. (2019). 
\newblock Flood \& flush: Low-cost security attacks on blockchain light clients. 
\newblock \textit{Proceedings of Financial Cryptography and Data Security}.

\bibitem{buterin2014next}
Buterin, V. (2014). 
\newblock A next-generation smart contract and decentralized application platform. 
\newblock \textit{Ethereum White Paper}.

\bibitem{kairouz2021advances}
Kairouz, P., McMahan, H. B., Avent, B., Bellet, A., Bennis, M., \ldots \& Zhou, S. (2021). 
\newblock Advances and open problems in federated learning. 
\newblock \textit{Foundations and Trends in Machine Learning}, 14(1-2), 1-210.

\bibitem{dwork2014algorithmic}
Dwork, C., \& Roth, A. (2014). 
\newblock The algorithmic foundations of differential privacy. 
\newblock \textit{Foundations and Trends in Theoretical Computer Science}, 9(3-4), 211-407.

\bibitem{bentov2014proof}
Bentov, I., Lee, C., Mizrahi, A., \& Rosenfeld, M. (2014). 
\newblock Proof of activity: Extending bitcoin's proof of work via proof of stake. 
\newblock \textit{Communications of the ACM}, 59(11), 76-85.

\bibitem{kwon2014tendermint}
Kwon, J. (2014). 
\newblock Tendermint: Consensus without mining. 
\newblock \textit{Draft version 0.1}.

\bibitem{wood2014ethereum}
Wood, G. (2014). 
\newblock Ethereum: A secure decentralised generalised transaction ledger. 
\newblock \textit{Ethereum Project Yellow Paper}, 151(2014), 1-32.

\bibitem{shoker2020decentralized}
Shoker, A. (2020). 
\newblock Decentralized threat intelligence: A new approach for a new era. 
\newblock \textit{IEEE Security \& Privacy}, 18(3), 58-65.

\bibitem{meiklejohn2019towards}
Meiklejohn, S., \& Hopper, N. (2019). 
\newblock Towards a methodology for collecting and analysing threat intelligence. 
\newblock \textit{Proceedings on Privacy Enhancing Technologies}, 2019(4), 229-248.

\bibitem{li2020federated}
Li, T. T., Sahu, A. K., Talwalkar, A., \& Smith, V. (2020). 
\newblock Federated learning: Challenges, methods, and future directions. 
\newblock \textit{IEEE Signal Processing Magazine}, 37(3), 50-60.

\bibitem{zyskind2015decentralizing}
Zyskind, G., Nathan, O., \& Pentland, A. (2015). 
\newblock Decentralizing privacy: Using blockchain to protect personal data. 
\newblock \textit{2015 IEEE Security and Privacy Workshops}, 180-184.

\bibitem{wang2019survey}
Wang, W., Hoang, D. T., Hu, P., Xiong, Z., Niyato, D., Wang, P., Wen, Y., \& Kim, D. I. (2019). 
\newblock A survey on consensus mechanisms and mining strategy management in blockchain networks. 
\newblock \textit{IEEE Access}, vol. 7, pp. 22328-22370. doi: 10.1109/ACCESS.2019.2896108.

\bibitem{yang2019federated}
Yang, Q., Liu, Y., Chen, T., \& Tong, Y. (2019). 
\newblock Federated machine learning: Concept and applications. 
\newblock \textit{ACM Transactions on Intelligent Systems and Technology}, 10(2), 1-19.

\end{thebibliography}

\end{document}