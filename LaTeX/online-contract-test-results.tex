\section{在线合约查询性能测试}

\subsection{1000 IP在线合约查询测试}

我们对OraSRS协议进行了在线合约查询性能测试,通过区块链合约接口查询1000个IP的风险评估。

\subsubsection{测试配置}
\begin{itemize}
\item 测试类型:1000 IP在线合约查询测试
\item 目标协议链:https://api.orasrs.net
\item 合约接口:OraSRSReader合约
\item 批次大小:10个IP/批次
\item 批次延迟:1秒/批次
\end{itemize}

\subsubsection{性能指标}
\begin{itemize}
\item 总处理时间:102.592秒
\item 成功查询数:1000
\item 失败查询数:0
\item 查询成功率:100\%
\item 平均处理时间:102.59ms/查询
\item 请求处理速度:9.75 RPS (Requests Per Second)
\item 平均风险评分:0.0000
\end{itemize}

\subsubsection{测试分析}

在线合约查询测试表明OraSRS协议链在批量IP风险评估方面具有良好的性能和稳定性。尽管平均处理时间相对较长(102.59ms/查询),但这是由于区块链网络的固有延迟特性所致。在实际应用中,客户端通常不会进行如此大规模的批量查询,而是进行单个或小批量查询。

测试中所有查询都成功完成(100\%成功率),表明合约接口稳定可靠。平均风险评分为0.0000,这符合预期,因为测试使用的是随机生成的IP地址,这些IP地址在协议链上没有相关的威胁记录。

\subsection{与本地测试对比}

我们将在线合约查询测试结果与之前的本地性能测试进行对比:

\begin{table}[H]
\centering
\begin{tabular}{@{}lccc@{}}
\toprule
\textbf{测试类型} & \textbf{平均处理时间} & \textbf{成功率} & \textbf{TPS} \\
\midrule
本地10000 IP测试 & 0.0376ms & 100\% & 26,595.74 \\
在线1000 IP合约测试 & 102.59ms & 100\% & 9.75 \\
\bottomrule
\end{tabular}
\caption{本地测试与在线合约测试性能对比}
\label{tab:local_vs_contract_performance}
\end{table}

正如预期,区块链合约查询的延迟明显高于本地测试,这是由于区块链网络通信、共识机制和分布式存储等因素造成的。然而,这种延迟在实际应用场景中是可以接受的,因为安全评估通常不是实时阻断决策,而是作为风险评估参考。