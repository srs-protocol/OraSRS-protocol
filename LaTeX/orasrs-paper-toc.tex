% OraSRS协议论文目录

\tableofcontents

\newpage

\section{引言} % 第1页
\subsection{研究背景}
\subsection{研究目标}
\subsection{研究意义}
\subsection{论文结构}

\section{相关工作} % 第2页
\subsection{传统威胁情报服务}
\subsection{区块链安全应用}
\subsection{联邦学习在安全中的应用}
\subsection{去中心化安全协议}

\section{OraSRS协议架构} % 第3页
\subsection{设计原则}
\subsubsection{咨询式服务模型}
\subsubsection{透明性与可审计性}
\subsubsection{隐私保护}
\subsubsection{合规性}
\subsection{三层架构设计}
\subsubsection{边缘层(Edge Layer)}
\subsubsection{共识层(Consensus Layer)}
\subsubsection{智能层(Intelligence Layer)}
\subsection{系统组成}

\section{核心算法与机制} % 第4页
\subsection{风险评分算法}
\subsubsection{多维度评分}
\subsubsection{时间衰减机制}
\subsubsection{权重调整策略}
\subsection{联邦学习机制}
\subsubsection{分布式训练}
\subsubsection{隐私保护}
\subsubsection{模型聚合}
\subsection{共识与质押机制}
\subsubsection{无质押注册}
\subsubsection{声誉系统}
\subsubsection{BFT共识算法}

\section{隐私保护与合规性} % 第5页
\subsection{数据最小化原则}
\subsection{隐私保护措施}
\subsubsection{IP匿名化处理}
\subsubsection{数据加密}
\subsubsection{访问控制}
\subsection{合规性设计}
\subsubsection{GDPR合规}
\subsubsection{CCPA合规}
\subsubsection{中国网络安全法合规}
\subsection{审计与透明性}

\section{智能合约设计} % 第6页
\subsection{威胁情报协调合约}
\subsubsection{数据结构设计}
\subsubsection{核心功能实现}
\subsubsection{事件与日志}
\subsection{批量处理合约}
\subsubsection{批量操作优化}
\subsubsection{性能提升机制}
\subsubsection{Gas费用优化}
\subsection{治理合约}
\subsubsection{参数治理}
\subsubsection{升级机制}
\subsubsection{紧急控制}

\section{性能优化与扩展性} % 第7页
\subsection{三层架构优化}
\subsubsection{边缘层优化}
\subsubsection{共识层优化}
\subsubsection{智能层优化}
\subsection{内核级威胁处理}
\subsubsection{ipset集成}
\subsubsection{O(1)匹配算法}
\subsubsection{自动超时清理}
\subsection{批量处理优化}
\subsubsection{合约批量操作}
\subsubsection{事件数量减少}
\subsubsection{10万级黑名单支持}

\section{实验与评估} % 第8页
\subsection{实验环境}
\subsubsection{硬件配置}
\subsubsection{软件环境}
\subsubsection{测试数据集}
\subsection{性能测试}
\subsubsection{响应时间测试}
\subsubsection{吞吐量测试}
\subsubsection{内存占用测试}
\subsection{准确性评估}
\subsubsection{误报率测试}
\subsubsection{漏报率测试}
\subsubsection{对比实验}
\subsection{安全性测试}

\section{安全性分析} % 第9页
\subsection{威胁模型}
\subsubsection{攻击者能力}
\subsubsection{安全目标}
\subsection{对抗攻击}
\subsubsection{垃圾信息攻击}
\subsubsection{女巫攻击}
\subsubsection{拜占庭故障}
\subsection{隐私保护分析}
\subsubsection{数据泄露防护}
\subsubsection{身份保护}
\subsubsection{差分隐私}

\section{部署与应用} % 第10页
\subsection{系统部署}
\subsubsection{一键部署方案}
\subsubsection{容器化部署}
\subsubsection{云原生部署}
\subsection{客户端应用}
\subsubsection{浏览器扩展}
\subsubsection{防火墙集成}
\subsubsection{API接口}
\subsection{生态系统}

\section{未来工作与结论} % 第11页
\subsection{技术演进方向}
\subsubsection{跨链集成}
\subsubsection{AI增强}
\subsubsection{零知识证明}
\subsection{生态扩展}
\subsubsection{合作伙伴集成}
\subsubsection{国际化部署}
\subsection{总结}
\subsubsection{主要贡献}
\subsubsection{创新点}
\subsubsection{局限性与展望}

\newpage

% 参考文献 % 第12页
\begin{thebibliography}{99}

\bibitem{nakamoto2008bitcoin}
Nakamoto, S. (2008). 
\newblock Bitcoin: A peer-to-peer electronic cash system.

\bibitem{mcmahan2017communication}
McMahan, B., Moore, E., Ramage, D., \& Yu, H. (2017). 
\newblock Communication-efficient learning of deep networks from decentralized data. 
\newblock \textit{Artificial Intelligence and Statistics}, 1273-1282.

\bibitem{goodell2019flood}
Goodell, G., Leiding, B., \& Johnson, H. (2019). 
\newblock Flood \& flush: Low-cost security attacks on blockchain light clients. 
\newblock \textit{Proceedings of Financial Cryptography and Data Security}.

\bibitem{buterin2014next}
Buterin, V. (2014). 
\newblock A next-generation smart contract and decentralized application platform. 
\newblock \textit{Ethereum White Paper}.

\bibitem{kairouz2021advances}
Kairouz, P., McMahan, H. B., Avent, B., Bellet, A., Bennis, M., \ldots \& Zhou, S. (2021). 
\newblock Advances and open problems in federated learning. 
\newblock \textit{Foundations and Trends in Machine Learning}, 14(1-2), 1-210.

\end{thebibliography}

\newpage

% 附录 % 第13页
\appendix
\section{智能合约代码}
\subsection{威胁情报协调合约}
\subsection{批量处理合约}
\subsection{治理合约}

\section{性能测试数据}
\subsection{基准测试结果}
\subsection{压力测试结果}
\subsection{稳定性测试结果}

\section{部署指南}
\subsection{环境准备}
\subsection{安装步骤}
\subsection{配置说明}

\section{API参考}
\subsection{查询接口}
\subsection{报告接口}
\subsection{管理接口}