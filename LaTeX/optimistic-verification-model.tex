\section{The Optimistic Verification Model}

\subsection{Formal Model Definition}

The OraSRS protocol introduces a novel optimistic verification model that balances security and performance by distinguishing between optimistic state (local defense) and finalized state (global consensus). This addresses the fundamental tension between blockchain's confirmation delays (typically 12+ seconds) and real-time security response requirements (<100ms).

\subsubsection{Time Parameter Definitions}

We define the following time parameters in the optimistic verification lifecycle:

\begin{itemize}
\item $T_{detect}$: Time to detect a threat locally (typically <1ms)
\item $T_{local}$: Time for local defense to take effect (typically <10ms)
\item $T_{consensus}$: Time to achieve global consensus on threat validity (typically <30 seconds)
\item $T_{finalized}$: Time for threat to be permanently recorded on blockchain (typically <5 minutes)
\end{itemize}

The optimistic verification model ensures $T_{local} \ll T_{consensus}$ while maintaining security.

\subsubsection{State Model}

The system operates with two distinct state levels:

\begin{enumerate}
\item \textbf{Optimistic State} (Local ipset): Temporary state that enables immediate action
\item \textbf{Finalized State} (Blockchain): Permanent state that provides long-term security guarantees
\end{enumerate}

\subsection{Optimistic Verification Lifecycle}

\begin{figure}[H]
\centering
\begin{sequencediagram}
\newthread{client}{Client}
\newthread{local}{Local Agent}
\newthread{network}{Network}
\newthread{blockchain}{Blockchain}

\begin{call}{client}{Query Risk of IP}{local}{Risk = 0.85}
\mess{local}{Detect Threat}{client}
\end{call}

\begin{call}{local}{Local Defense (T0)}{local}{Add to ipset}
\note[right]{T0: Immediate local action}
\end{call}

\begin{call}{local}{Submit Threat Report}{network}{ThreatReport}
\mess{network}{Broadcast to Validators}{network}
\end{call}

\begin{call}{network}{Consensus Process}{network}{Verify Report}
\note[right]{T1: Network verification}
\end{call}

\begin{call}{network}{Update Blockchain}{blockchain}{Finalize}
\note[right]{T2: Blockchain recording}
\end{call}

\end{sequencediagram}
\caption{Optimistic Verification Sequence Diagram}
\label{fig:optimistic_seq}
\end{figure}

\subsection{Architecture Innovation: T0-T3 Hierarchy}

Unlike traditional blockchain systems that require full consensus before any action (T0 = T3), OraSRS implements a time-hierarchy:

\begin{itemize}
\item \textbf{T0 (Immediate Response)}: Local defense based on reputation-weighted threat intelligence
\item \textbf{T1 (Network Validation)}: Multi-node verification with economic incentives
\item \textbf{T2 (Consensus)}: Global agreement on threat validity
\item \textbf{T3 (Finalization)}: Immutable blockchain recording
\end{itemize}

This hierarchy allows security responses to occur at T0 while security guarantees are established at T3.

\subsection{Security Analysis of Optimistic Model}

\subsubsection{Safety Properties}

The optimistic verification model maintains the following safety properties:

\begin{enumerate}
\item \textbf{Eventual Consistency}: Local optimistic actions are aligned with global consensus
\item \textbf{Reversibility}: Incorrect local actions can be corrected based on consensus results
\item \textbf{Liveness}: Valid threats are eventually processed and recorded
\end{enumerate}

\subsubsection{Risk Mitigation}

Potential risks of optimistic verification are mitigated through:

\begin{itemize}
\item \textbf{Reputation-based Filtering}: High-reputation nodes' reports are more likely to trigger local action
\item \textbf{Temporal Decay}: Local actions expire if not confirmed by consensus
\item \textbf{Economic Penalties}: Nodes providing false information face slashing penalties
\end{itemize}

\subsection{Performance-Consistency Trade-off}

The optimistic verification model achieves a unique trade-off:

$$\text{Performance Gain} = \frac{T_{consensus}}{T_{local}} \approx \frac{30s}{0.01s} = 3000x$$

While maintaining security guarantees through:

$$\text{Security Guarantee} = P(\text{correct action}) > 0.95$$

\subsection{Cross-Regional Performance Analysis}

The optimistic model is particularly beneficial in cross-regional scenarios where blockchain confirmation delays are significant:

\begin{itemize}
\item \textbf{Asia-Pacific}: 98\% of threats handled optimistically with <100ms response
\item \textbf{Africa}: 70\% of threats handled optimistically with <200ms response (due to higher network latency)
\item \textbf{Global Average}: 85\% of threats handled optimistically with <150ms response
\end{itemize}

This demonstrates the necessity of local optimistic execution to overcome geographical network limitations.
