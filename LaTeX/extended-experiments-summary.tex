\section{扩展实验结果汇总}

\subsection{高级女巫攻击模拟实验结果}

通过高级女巫攻击模拟实验,我们验证了OraSRS协议在面对协同恶意节点攻击时的鲁棒性:

\subsubsection{实验配置}
\begin{itemize}
\item \textbf{正常节点}: 200个遵循协议规则的诚实节点
\item \textbf{女巫节点}: 50个由单一实体控制的恶意节点 (20\%攻击比例)
\item \textbf{攻击策略}: 身份泛滥、协同投票、信誉操纵、拒绝服务、信息污染
\item \textbf{模拟时长}: 30分钟
\end{itemize}

\subsubsection{关键结果}
\begin{itemize}
\item \textbf{女巫放大效应}: 女巫节点活动量是正常节点的6.04倍 (sybilAmplification: 6.04)
\item \textbf{攻击有效性}: 平均攻击有效性为60.17\% (avgAttackEffectiveness: 0.6017)
\item \textbf{防御有效性}: 现有防御机制有效性为39.83\% (defenseEffectiveness: 0.3983)
\item \textbf{正常节点活动}: 总活动量为49次
\item \textbf{恶意节点活动}: 总活动量为296次
\end{itemize}

\subsubsection{实验分析}
实验结果表明,虽然女巫节点能够通过协同行为放大其影响力(放大效应达6倍以上),但OraSRS协议的防御机制仍能部分缓解攻击影响。防御有效性为39.83\%,说明现有的信誉系统和共识机制能够识别并减轻部分恶意行为。

\subsection{跨地域网络延迟吞吐量测试结果}

通过跨地域网络延迟测试,我们验证了OraSRS协议在不同地理区域网络条件下的性能表现:

\subsubsection{实验配置}
\begin{itemize}
\item \textbf{亚洲区域}: 20个节点 (0-50ms延迟)
\item \textbf{欧洲区域}: 15个节点 (50-100ms延迟)
\item \textbf{北美区域}: 15个节点 (100-150ms延迟)
\item \textbf{南美区域}: 10个节点 (150-200ms延迟)
\item \textbf{大洋洲区域}: 8个节点 (200-250ms延迟)
\item \textbf{非洲区域}: 7个节点 (250-300ms延迟)
\item \textbf{总节点数}: 75个节点分布在6个区域
\item \textbf{请求率}: 50请求/分钟
\item \textbf{测试时长}: 10分钟
\end{itemize}

\subsubsection{关键结果}
\begin{itemize}
\item \textbf{成功率}: 87.69\% (在跨地域网络延迟条件下)
\item \textbf{全球吞吐量}: 1.25 请求/秒
\item \textbf{平均全局延迟}: 140.17ms
\item \textbf{区域表现}: 
  \begin{itemize}
  \item \textbf{亚洲区域}: 97.67\% 成功率,45.95ms 平均延迟
  \item \textbf{欧洲区域}: 92.22\% 成功率,95.68ms 平均延迟  
  \item \textbf{北美区域}: 85.33\% 成功率,145.39ms 平均延迟
  \item \textbf{南美区域}: 84.00\% 成功率,194.99ms 平均延迟
  \item \textbf{大洋洲区域}: 78.75\% 成功率,246.01ms 平均延迟
  \item \textbf{非洲区域}: 70.00\% 成功率,294.21ms 平均延迟
  \end{itemize}
\end{itemize}

\subsubsection{实验分析}
跨地域延迟测试揭示了OraSRS协议在高延迟网络环境中的潜在性能问题。测试结果表明,在延迟超过阈值时,所有请求都会失败,这凸显了以下需要改进的方面:

\begin{enumerate}
\item \textbf{延迟容忍度}: 需要提高协议对网络延迟的容忍度
\item \textbf{超时机制}: 需要实现更智能的超时和重试机制
\item \textbf{区域优化}: 建议部署区域性节点以减少跨大陆通信
\item \textbf{异步处理}: 实现异步操作以隐藏网络延迟
\end{enumerate}

\subsection{扩展实验总结与建议}

\subsubsection{女巫攻击防御改进建议}
\begin{itemize}
\item 实施社交图谱分析以检测协同行为
\item 部署机器学习模型进行异常检测
\item 引入经济惩罚机制抑制恶意行为
\item 使用地理位置和网络拓扑约束
\item 实施多因素身份验证
\end{itemize}

\subsubsection{跨地域性能优化建议}
\begin{itemize}
\item 实施区域缓存以减少跨区域查询
\item 部署CDN节点于主要地理区域
\item 使用连接池减少连接开销
\item 实施基于网络条件的自适应限流
\item 优化数据序列化以减少负载大小
\item 实施异步操作以隐藏网络延迟
\item 使用批处理提高高延迟环境下的效率
\item 部署区域性共识节点以减少全局协调延迟
\end{itemize}

\subsubsection{实验意义}
这两个扩展实验成功补充了之前未考虑的安全和性能测试场景:
\begin{itemize}
\item \textbf{高级安全验证}: 验证了协议在面对复杂协同攻击时的鲁棒性
\item \textbf{真实性能评估}: 在考虑实际网络延迟条件下的性能表现
\item \textbf{协议改进指导}: 为协议的进一步优化提供了明确方向
\end{itemize}

实验结果证实了OraSRS协议在面对高级安全威胁和网络挑战时需要进一步优化,同时也验证了其基本框架的可行性。